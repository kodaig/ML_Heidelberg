\documentclass[12pt]{scrartcl}

% !TeX root = ./main.tex

% write thesis using scrreprt class, see also
% http://mirrors.ctan.org/macros/latex/contrib/koma-script/scrguien.pdf or
% http://mirrors.ctan.org/macros/latex/contrib/koma-script/scrguide.pdf
% \documentclass[paper=a4,bibliography=totoc,abstract=on,10pt]{scrartcl}

\usepackage{
    amsmath,
    graphicx,
    % url,
    tabularx,
    % booktabs,
    % multirow,
    % epstopdf,
    caption,
    subcaption,
    float,
    % floatrow,
    listingsutf8,
}

\usepackage[utf8]{inputenc}
\usepackage[english]{babel}

\usepackage[load-configurations = abbreviations]{siunitx}
\sisetup{
	separate-uncertainty,
	% table-number-alignment = center,
	% table-sign-mantissa,
	exponent-product = \cdot,
	% tophrase={{\ bis\ }}
}

% \usepackage[bookmarks=true,unicode=true]{hyperref}
% \usepackage[english]{cleveref}

% \linespread{1.25}
\setlength{\parindent}{0pt} %no indent

\captionsetup{format=plain,font=small,margin=20pt}

\lstset{
  language=python,
  % backgroundcolor=\color[RGB]{245,245,244},
  % otherkeywords={self},             % Add keywords here
  % keywordstyle=\color{blue},
  % commentstyle=\it\color[RGB]{0,96,96}\ttfamily,
  % stringstyle=\color[RGB]{255,140,0},
  % numbers=left,
  % stepnumber=5,
  showspaces=false,
  showstringspaces=false,
  frame=single, %tb
  breaklines=true
}

% \graphicspath{{pictures}}

% \floatsetup[table]{capposition= top,objectset=centering}
% \floatsetup[figure]{objectset=centering}



\author{Kodai Matsuoka, Yuyan Li}
\subject{Machine Learning for Computer Vision}
\title{Exercise 5}
% \subtitle{}
\date{26 May, 2017}


\begin{document}

\maketitle

\section{Features}

Our unary and Potts features are computed with the code in \Cref{lst:feat}.

\lstinputlisting[caption=Feature functions,linerange={30-63},label=lst:feat]{StructLearn.py}

\section{Test set performance with GraphCut}
The full code is at the bottom in \Cref{lst:full}. We have a variable \code{mode} for the different exercises.

For this part we chose \code{mode='gp'} to use graph cut.

The computed loss values for different noises and regularizers are:

\begin{verbatim}
[[[ 7 15  1 10  8]
  [ 7 13  2  8  8]
  [ 7 13  2  9  9]
  [ 7 14  1 10  9]
  [ 7 14  1 10  9]]

 [[16 14 17 15 19]
  [15 15 17 19 20]
  [16 15 12 19 20]
  [16 16 13 20 20]
  [16 15 13 20 20]]

 [[10 12 16 10 11]
  [17 14 22 10 15]
  [17 14 23 10 15]
  [17 14 22 10 15]
  [19 14 22 10 15]]

 [[22 18 16 21 15]
  [20 21 19 24 16]
  [20 23 19 22 18]
  [20 23 19 24 18]
  [20 23 19 24 18]]

 [[40 25 18 32 22]
  [40 30 18 41 32]
  [37 28 18 40 38]
  [41 30 17 42 43]
  [41 30 17 42 43]]]
\end{verbatim}

The content of this list structure is:

- every row is the loss of the five test images of a certain regularizer C and noise

- every list of five rows belong to a certain noise

The noises and regularizer values are sorted as given in the exercise.

As expected, a bigger noise leads to worse predictions and therefore higher losses.

\section{Test set performance with ICM}

The same is done using an ICM solver instead of GraphCut by setting \code{mode='icm'}.

\begin{verbatim}
[[[ 5  5  6  9  5]
  [ 5  5  6  9  5]
  [ 5  5  6  8  5]
  [ 5  5  6  8  5]
  [ 5  5  6  8  5]]

 [[ 1 10 12 14  0]
  [ 5 11 12 15  0]
  [ 4 12 11 14  0]
  [ 5 11 13 15  0]
  [ 5 11 13 14  0]]

 [[16 12 12 17 15]
  [17 13 22 18 14]
  [17 13 21 19 14]
  [17 13 22 20 14]
  [17 13 22 20 14]]

 [[21 22 22 17 16]
  [22 25 20 22 17]
  [23 27 20 22 16]
  [22 24 19 22 16]
  [22 24 18 22 16]]

 [[56 46 21 26 34]
  [40 24 34 14 60]
  [46 24 43 17 60]
  [46 24 55 19 60]
  [44 26 44 21 60]]]
\end{verbatim}

Qualitatively, there does not seem to be a difference to the GraphCut results.

\section{Bonus: show performance during training}

With the code in \Cref{lst:bonus} (part of \code{subgradient\_ssvm})the training and test set performances are show after every iteration during training.

If \code{mode='bonus'} then after every training a plot with the losses is shown. One exemplary plot can be seen in \Cref{fig:bonus}.

\begin{figure}
    \centering
    \includegraphics[width=\linewidth]{loss}
    \label{fig:bonus}
\end{figure}

\lstinputlisting[caption=Code to show performance after every iteration during training.,linerange={243-273},label=lst:bonus]{StructLearn.py}



\clearpage
\lstinputlisting[caption=StructLearn.py,label=lst:full]{StructLearn.py}

\end{document}

