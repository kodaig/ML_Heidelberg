\documentclass[12pt]{scrartcl}

%%%%%
% Header for solutions for the course Machine Learning for Computer Vision
% Summer 2017
%%%%%

\usepackage[utf8]{inputenc}
\usepackage[english]{babel}

\usepackage{
  amsmath,
  graphicx,
  tabularx,
  caption,
  subcaption,
  float,
  listingsutf8,
}

\usepackage[load-configurations = abbreviations]{siunitx}
\usepackage{hyperref}
\usepackage[english]{cleveref}

%%%%% format settings
%------------
% \setlength{\parindent}{0pt} %no indent
%------------

%%%%% settings for listings
\lstset{
  language=python,
  basicstyle=\footnotesize,  % font size
  showspaces=false,
  showstringspaces=false,
  frame=single, %tb
  breaklines=true,
  % backgroundcolor=\color[RGB]{245,245,244},
  % otherkeywords={self},             % Add keywords here
  % keywordstyle=\color{blue},
  % commentstyle=\it\color[RGB]{0,96,96}\ttfamily,
  % stringstyle=\color[RGB]{255,140,0},
  % numbers=left,
  % stepnumber=5,
}

%%% commands
%------------
\newcommand{\code}{\texttt}
%------------


\author{Kodai Matsuoka, Yuyan Li}
\subject{Machine Learning for Computer Vision}
\title{Exercise 7}
% \subtitle{}
\date{June 12, 2017}


\begin{document}

\maketitle

\section{Implementation}

Our implementation of the Gaussian Graphical Model (GGM) is shown below in \cref{lst:den}.


\section{Comparison}

We compared our algorithm to some of the algorithms available on ipol.im.

In particular we used the \emph{Non-Local Means Denoising} to do a qualitative comparision. For that we used their program to first produce a noisy image which we denoised with both their and our algorithm. Then we used their program \emph{img\_mse\_ipol} to calculate the RMSE and the PSNR. The better denoising has a smaller RMSE and larger PSNR value. We can see that the GGM is still far from achieving the results of the \emph{nlmeans} algorithm.

\begin{table}
  \centering
  \caption{RMSE and PSNR values of different algorithm with noise value $\sigma=30$.}
  \begin{tabular}{l | r | r}
          & GGM   & nlmeans \\
    \hline
    RMSE  & 13.66 & 8.00    \\
    PSNR  & 25.42 & 30.07
  \end{tabular}
  \label{tab:rmse}
\end{table}

\begin{figure}
  \centering
  \includegraphics[width=.3\linewidth]{compare/noisy}
  \includegraphics[width=.3\linewidth]{compare/nldenoised}
  \includegraphics[width=.3\linewidth]{compare/mydenoised}
  \label{fig:den}
  \caption{From left to right the pictures show: the noisy image, denoised with \emph{nlmeans}, denoised with our algorithm.}
\end{figure}

\begin{figure}
  \centering
  \includegraphics[width=.3\linewidth]{compare/nldifference}
  \includegraphics[width=.3\linewidth]{compare/mydifference}
  \label{fig:diff}
  \caption{The differences to the original image. On the left is \emph{nlmeans} and on the right our algorithm.}
\end{figure}




\clearpage
\lstinputlisting[caption=denoising.py,label=lst:den]{denoising.py}

\end{document}


