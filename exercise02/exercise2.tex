\documentclass[12pt,a4paper]{scrartcl}

%%%%%
% Header for solutions for the course Machine Learning for Computer Vision
% Summer 2017
%%%%%

\usepackage[utf8]{inputenc}
\usepackage[english]{babel}

\usepackage{
  amsmath,
  graphicx,
  tabularx,
  caption,
  subcaption,
  float,
  listingsutf8,
}

\usepackage[load-configurations = abbreviations]{siunitx}
\usepackage{hyperref}
\usepackage[english]{cleveref}

%%%%% format settings
%------------
% \setlength{\parindent}{0pt} %no indent
%------------

%%%%% settings for listings
\lstset{
  language=python,
  basicstyle=\footnotesize,  % font size
  showspaces=false,
  showstringspaces=false,
  frame=single, %tb
  breaklines=true,
  % backgroundcolor=\color[RGB]{245,245,244},
  % otherkeywords={self},             % Add keywords here
  % keywordstyle=\color{blue},
  % commentstyle=\it\color[RGB]{0,96,96}\ttfamily,
  % stringstyle=\color[RGB]{255,140,0},
  % numbers=left,
  % stepnumber=5,
}

%%% commands
%------------
\newcommand{\code}{\texttt}
%------------


\author{Kodai Matsuoke, Yuyan Li}
\subject{Machine Learning for Computer Vision}
\title{Exercise 2}
% \subtitle{}
\date{May 5, 2017}

\usepackage{expl3}
\ExplSyntaxOn
\cs_new_eq:NN \Repeat \prg_replicate:nn
\ExplSyntaxOff

\newcommand{\infrow}[1]{\multicolumn{#1}{c}{$\infty$}}

\begin{document}

\maketitle

\section{Higher order factors}
The domain of $x_z$ is ${0,1,2,3,4,5,6,7}$. Each variable value represents one energy state. The pairwise factors are given in the following table:

\begin{tabular}{r|cccccccccccccccc}
  \hline
  $x_z$        & 0 & 1 & 2 & 3 & 4 & 5 & 6 & 7 & 0 & 1 & 2 & 3 & 4 & 5 & 6 & 7 \\
  $x_0$        & 0 & 0 & 0 & 0 & 0 & 0 & 0 & 0 & 1 & 1 & 1 & 1 & 1 & 1 & 1 & 1 \\
  \hline
  $\phi_{0z}$  & a & b & c & d & \infrow{4}    & \infrow{4}    & e & f & g & h \\
  \hline
\end{tabular}
\\
\begin{tabular}{r|cccccccccccccccc}
  \hline
  $x_z$        & 0 & 1 & 2 & 3 & 4 & 5 & 6 & 7 & 0 & 1 & 2 & 3 & 4 & 5 & 6 & 7 \\
  $x_1$        & 0 & 0 & 0 & 0 & 0 & 0 & 0 & 0 & 1 & 1 & 1 & 1 & 1 & 1 & 1 & 1 \\
  \hline
  $\phi_{1z}$  & 0 & 0 & \infrow{2} & 0 & 0 & \infrow{2} & \infrow{2} & 0 & 0 & \infrow{2} & 0 & 0 \\
  \hline
\end{tabular}

\begin{tabular}{r|cccccccccccccccc}
  \hline
  $x_z$        & 0 & 1 & 2 & 3 & 4 & 5 & 6 & 7 & 0 & 1 & 2 & 3 & 4 & 5 & 6 & 7 \\
  $x_1$        & 0 & 0 & 0 & 0 & 0 & 0 & 0 & 0 & 1 & 1 & 1 & 1 & 1 & 1 & 1 & 1 \\
  \hline
  $\phi_{1z}$  \Repeat{8}{& 0 & $\infty$} \\
  \hline
\end{tabular}

\end{document}


