\documentclass[12pt,a4paper]{scrartcl}

%%%%%
% Header for solutions for the course Machine Learning for Computer Vision
% Summer 2017
%%%%%

\usepackage[utf8]{inputenc}
\usepackage[english]{babel}

\usepackage{
  amsmath,
  graphicx,
  tabularx,
  caption,
  subcaption,
  float,
  listingsutf8,
}

\usepackage[load-configurations = abbreviations]{siunitx}
\usepackage{hyperref}
\usepackage[english]{cleveref}

%%%%% format settings
%------------
% \setlength{\parindent}{0pt} %no indent
%------------

%%%%% settings for listings
\lstset{
  language=python,
  basicstyle=\footnotesize,  % font size
  showspaces=false,
  showstringspaces=false,
  frame=single, %tb
  breaklines=true,
  % backgroundcolor=\color[RGB]{245,245,244},
  % otherkeywords={self},             % Add keywords here
  % keywordstyle=\color{blue},
  % commentstyle=\it\color[RGB]{0,96,96}\ttfamily,
  % stringstyle=\color[RGB]{255,140,0},
  % numbers=left,
  % stepnumber=5,
}

%%% commands
%------------
\newcommand{\code}{\texttt}
%------------


\author{Kodai Matsuoka, Yuyan Li}
\subject{Machine Learning for Computer Vision}
\title{Exercise 2}
% \subtitle{}
\date{May 5, 2017}

\usepackage{expl3}
\ExplSyntaxOn
\cs_new_eq:NN \Repeat \prg_replicate:nn
\ExplSyntaxOff

\newcommand{\infrow}[1]{\multicolumn{#1}{c}{$\infty$}}

\begin{document}

\maketitle

\section{Iterated Conditional Models}

The missing code is:

\begin{verbatim}
# unary terms
energy += - beta * math.log(unaries[x0,x1,l])

# pairwise terms
energy += 4 - [labels[x0-1,x1], labels[x0+1,x1],
            labels[x0,x1-1], labels[x0,x1+1]].count(l)
\end{verbatim}

The regularizer \textit{beta} changes the coarseness of the labeling.

The code to use probability pictures as unaries is:

\begin{verbatim}
# import predictions from exercise1
# prediction images are in folder predictions/
pred_paths = glob.glob("predictions/*")
pred = [skimage.img_as_float(skimage.io.imread(f)) for f in pred_paths]

# Getting rid of the zeros
for x in numpy.nditer(pred[0], op_flags=['readwrite']):
    if x == 0:
        x[...] = 1e-100
    if x == 1:
        x[...] = 1. - 1e-16

fg = p
bg = 1.-p
unaries = numpy.dstack((fg, bg))
\end{verbatim}

In the whole program (\textit{icm.py}) there is also an addition at the end to produce pictures of the labels. A few examples are shown here:

\includegraphics[width=.45\linewidth]{label0.png}
\includegraphics[width=.45\linewidth]{label1.png}
\\
\includegraphics[width=.45\linewidth]{label2.png}
\includegraphics[width=.45\linewidth]{label3.png}
\\
% \lstinputlisting[caption=icm.py,label=lst:icm]{icm.py}

\section{Higher order factors}
The domain of $x_z$ is $\{0,1,2,3,4,5,6,7\}$. Each variable value represents one energy state. The pairwise factors are given in the following table:


\begin{tabular}{r|cccccccccccccccc}
  \hline
  $x_z$        & 0 & 1 & 2 & 3 & 4 & 5 & 6 & 7 & 0 & 1 & 2 & 3 & 4 & 5 & 6 & 7 \\
  $x_0$        & 0 & 0 & 0 & 0 & 0 & 0 & 0 & 0 & 1 & 1 & 1 & 1 & 1 & 1 & 1 & 1 \\
  \hline
  $\phi_{0z}$  & a & b & c & d & \infrow{4}    & \infrow{4}    & e & f & g & h \\
  \hline
  \hline
  $x_z$        & 0 & 1 & 2 & 3 & 4 & 5 & 6 & 7 & 0 & 1 & 2 & 3 & 4 & 5 & 6 & 7 \\
  $x_1$        & 0 & 0 & 0 & 0 & 0 & 0 & 0 & 0 & 1 & 1 & 1 & 1 & 1 & 1 & 1 & 1 \\
  \hline
  $\phi_{1z}$  & 0 & 0 & \infrow{2} & 0 & 0 & \infrow{2} & \infrow{2} & 0 & 0 & \infrow{2} & 0 & 0 \\
  \hline
  \hline
  $x_z$        & 0 & 1 & 2 & 3 & 4 & 5 & 6 & 7 & 0 & 1 & 2 & 3 & 4 & 5 & 6 & 7 \\
  $x_1$        & 0 & 0 & 0 & 0 & 0 & 0 & 0 & 0 & 1 & 1 & 1 & 1 & 1 & 1 & 1 & 1 \\
  \hline
  $\phi_{1z}$  \Repeat{4}{& 0 & $\infty$} \Repeat{4}{& $\infty$ & 0}\\
  \hline
\end{tabular}

By using infinity in the pairwise factors, for any value for $x_z$ there is only one value that each $x_i$ can have which correspond with the energy given by $\phi_{012}$.


\end{document}


