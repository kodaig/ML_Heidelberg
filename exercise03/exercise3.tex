\documentclass[12pt,a4paper]{scrartcl}

%%%%%
% Header for solutions for the course Machine Learning for Computer Vision
% Summer 2017
%%%%%

\usepackage[utf8]{inputenc}
\usepackage[english]{babel}

\usepackage{
  amsmath,
  graphicx,
  tabularx,
  caption,
  subcaption,
  float,
  listingsutf8,
}

\usepackage[load-configurations = abbreviations]{siunitx}
\usepackage{hyperref}
\usepackage[english]{cleveref}

%%%%% format settings
%------------
% \setlength{\parindent}{0pt} %no indent
%------------

%%%%% settings for listings
\lstset{
  language=python,
  basicstyle=\footnotesize,  % font size
  showspaces=false,
  showstringspaces=false,
  frame=single, %tb
  breaklines=true,
  % backgroundcolor=\color[RGB]{245,245,244},
  % otherkeywords={self},             % Add keywords here
  % keywordstyle=\color{blue},
  % commentstyle=\it\color[RGB]{0,96,96}\ttfamily,
  % stringstyle=\color[RGB]{255,140,0},
  % numbers=left,
  % stepnumber=5,
}

%%% commands
%------------
\newcommand{\code}{\texttt}
%------------

\setcounter{MaxMatrixCols}{20}

\author{Kodai Matsuoka, Yuyan Li}
\subject{Machine Learning for Computer Vision}
\title{Exercise 3}
% \subtitle{}
\date{May 12, 2017}


\begin{document}

\maketitle

Our program is shown in \cref{lst:lp}.

For the Linear Program we have to use the vector $\mu$ as introduced in the lecture. The function \textit{givemu(x0,x1,x2)} produces such a vector for given (x0,x1,x2). For this system $\mu$ is a 18-component vector of the form:
\[
\mu = 
\begin{array}{ccccccccccc}
  ( & \mu_{0}(0) & \mu_{0}(1) & \mu_{1}(0) & \mu_{1}(1) & \mu_{2}(0) & \mu_{2}(1)
      & \mu_{01}(0,0) & \mu_{01}(0,1) & \mu_{01}(1,0) & \\
   & \mu_{01}(1,1) & \mu_{12}(0,0) & \mu_{12}(0,1) & \mu_{12}(1,0) & \mu_{12}(1,1)
      & \mu_{20}(0,0) & \mu_{20}(0,1) & \mu_{20}(1,0) & \mu_{20}(1,1) & )
\end{array}
\]

The function \textit{coeff(p0,p1,p2,p01,p02,p12)} takes the potentials $\psi_i$ and $\psi_{ij}$ and puts them into a coefficient vector. It is also a 18-component vector of the form:
\[
c = 
(
\begin{array}{cccccccccc}
\psi_{0}(0) & \psi_{0}(1) & \psi_{1}(0) & \psi_{1}(1) & \psi_{2}(0) & \psi_{2}(1) &
\psi_{01}(0,0) & \psi_{01}(0,1) & \psi_{01}(1,0) & \psi_{01}(1,1)
\end{array}
\]
\[
\begin{array}{cccccccc}
\psi_{12}(0,0) & \psi_{12}(0,1) & \psi_{12}(1,0) & \psi_{12}(1,1)
& \psi_{20}(0,0) & \psi_{20}(0,1) & \psi_{20}(1,0) & \psi_{20}(1,1)
\end{array}
)
\]

The constraint matrix A is chosen as:
\begin{equation}
   A =
   \begin{pmatrix}
   1 & 1 & 0 & 0 & 0 & 0 & 0 & 0 & 0 & 0 & 0 & 0 & 0 & 0 & 0 & 0 & 0 & 0 \\
   0 & 0 & 1 & 1 & 0 & 0 & 0 & 0 & 0 & 0 & 0 & 0 & 0 & 0 & 0 & 0 & 0 & 0 \\
   0 & 0 & 0 & 0 & 1 & 1 & 0 & 0 & 0 & 0 & 0 & 0 & 0 & 0 & 0 & 0 & 0 & 0 \\
   -1 & 0 & 0 & 0 & 0 & 0 & 1 & 1 & 0 & 0 & 0 & 0 & 0 & 0 & 0 & 0 & 0 & 0 \\
   0 & -1 & 0 & 0 & 0 & 0 & 0 & 0 & 1 & 1 & 0 & 0 & 0 & 0 & 0 & 0 & 0 & 0 \\
   0 & 0 & -1 & 0 & 0 & 0 & 1 & 0 & 1 & 0 & 0 & 0 & 0 & 0 & 0 & 0 & 0 & 0 \\
   0 & 0 & 0 & -1 & 0 & 0 & 0 & 1 & 0 & 1 & 0 & 0 & 0 & 0 & 0 & 0 & 0 & 0 \\
   0 & 0 & -1 & 0 & 0 & 0 & 0 & 0 & 0 & 0 & 1 & 1 & 0 & 0 & 0 & 0 & 0 & 0 \\
   0 & 0 & 0 & -1 & 0 & 0 & 0 & 0 & 0 & 0 & 0 & 0 & 1 & 1 & 0 & 0 & 0 & 0 \\
   0 & 0 & 0 & 0 & -1 & 0 & 0 & 0 & 0 & 0 & 1 & 0 & 1 & 0 & 0 & 0 & 0 & 0 \\
   0 & 0 & 0 & 0 & 0 & -1 & 0 & 0 & 0 & 0 & 0 & 1 & 0 & 1 & 0 & 0 & 0 & 0 \\
   0 & 0 & 0 & 0 & -1 & 0 & 0 & 0 & 0 & 0 & 0 & 0 & 0 & 0 & 1 & 1 & 0 & 0 \\
   0 & 0 & 0 & 0 & 0 & -1 & 0 & 0 & 0 & 0 & 0 & 0 & 0 & 0 & 0 & 0 & 1 & 1 \\
   -1 & 0 & 0 & 0 & 0 & 0 & 0 & 0 & 0 & 0 & 0 & 0 & 0 & 0 & 1 & 0 & 1 & 0 \\
   0 & -1 & 0 & 0 & 0 & 0 & 0 & 0 & 0 & 0 & 0 & 0 & 0 & 0 & 0 & 1 & 0 & 1
   \end{pmatrix}
\end{equation}
with the constraint (15-component) vector:
\[
  b = 
  \begin{pmatrix}
    1 \\
    1 \\
    1 \\
    0 \\
    \vdots \\
    0
  \end{pmatrix}
\]

The constraint is thus:
\[
  A \mu = b
\]

We use the given potentials to calculate $\mu$ for $\beta \in \{-1,1\}$.

The output for \textit{beta} = +1.0
\begin{verbatim}
  Optimization terminated successfully.
        Current function value: 1.100000    
        Iterations: 13
  beta= 1.0
  solution vector mu=
  [ 1.  0.  1.  0.  1.  0.  1.  0.  0.  0.  1.  0.  0.  0.  1.  0.  0.  0.]
  result: [ 0.  0.  0.]
\end{verbatim}

The output for \textit{beta} = -1.0
\begin{verbatim}
  Optimization terminated successfully.
          Current function value: -1.900000   
          Iterations: 16
  beta= -1.0
  solution vector mu=
  [ 0.5  0.5  0.5  0.5  0.5  0.5  0.   0.5  0.5  0.   0.   0.5  0.5  0.   0.
  0.5  0.5  0. ]
  result: [ 0.  0.  0.]
\end{verbatim}

Comment


1. We see that the program gives a non-integer result for $\beta=-1$. We need to use ILP to get the right solution.

2. For \(\beta=1\), the exact solutions (calculated by hand) are (x0,x1,x2)=(0,0,0) and (1,1,1). However it gave only one of these solutions which is (1,1,1). It also goes for \(\beta=-1\).

\clearpage
\lstinputlisting[caption=exercise3.py,label=lst:lp]{exercise3.py}

\end{document}

