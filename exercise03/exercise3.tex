\documentclass[12pt,a4paper]{scrartcl}

% !TeX root = ./main.tex

% write thesis using scrreprt class, see also
% http://mirrors.ctan.org/macros/latex/contrib/koma-script/scrguien.pdf or
% http://mirrors.ctan.org/macros/latex/contrib/koma-script/scrguide.pdf
% \documentclass[paper=a4,bibliography=totoc,abstract=on,10pt]{scrartcl}

\usepackage{
    amsmath,
    graphicx,
    % url,
    tabularx,
    % booktabs,
    % multirow,
    % epstopdf,
    caption,
    subcaption,
    float,
    % floatrow,
    listingsutf8,
}

\usepackage[utf8]{inputenc}
\usepackage[english]{babel}

\usepackage[load-configurations = abbreviations]{siunitx}
\sisetup{
	separate-uncertainty,
	% table-number-alignment = center,
	% table-sign-mantissa,
	exponent-product = \cdot,
	% tophrase={{\ bis\ }}
}

% \usepackage[bookmarks=true,unicode=true]{hyperref}
% \usepackage[english]{cleveref}

% \linespread{1.25}
\setlength{\parindent}{0pt} %no indent

\captionsetup{format=plain,font=small,margin=20pt}

\lstset{
  language=python,
  % backgroundcolor=\color[RGB]{245,245,244},
  % otherkeywords={self},             % Add keywords here
  % keywordstyle=\color{blue},
  % commentstyle=\it\color[RGB]{0,96,96}\ttfamily,
  % stringstyle=\color[RGB]{255,140,0},
  % numbers=left,
  % stepnumber=5,
  showspaces=false,
  showstringspaces=false,
  frame=single, %tb
  breaklines=true
}

% \graphicspath{{pictures}}

% \floatsetup[table]{capposition= top,objectset=centering}
% \floatsetup[figure]{objectset=centering}


\setcounter{MaxMatrixCols}{20}

\author{Kodai Matsuoke, Yuyan Li}
\subject{Machine Learning for Computer Vision}
\title{Exercise 3}
% \subtitle{}
\date{May 12, 2017}


\begin{document}

\maketitle

Our program is shown in \cref{lst:lp}.

For the Linear Program we have to use the vector $\mu$ as introduced in the lecture. For this system it is a 12-component vector.

The function \textit{coeff(p0,p1,p2,p01,p02,p12)} take the potentials $\psi_i$ and $\psi_{ij}$ and puts them into a coefficient vector. It is also a 12-component vector.

The constraint matrix A is chosen as:
\begin{equation}
   A =
   \begin{pmatrix}
   1 & 1 & 0 & 0 & 0 & 0 & 0 & 0 & 0 & 0 & 0 & 0 & 0 & 0 & 0 & 0 & 0 & 0 \\
   0 & 0 & 1 & 1 & 0 & 0 & 0 & 0 & 0 & 0 & 0 & 0 & 0 & 0 & 0 & 0 & 0 & 0 \\
   0 & 0 & 0 & 0 & 1 & 1 & 0 & 0 & 0 & 0 & 0 & 0 & 0 & 0 & 0 & 0 & 0 & 0 \\
   -1 & 0 & 0 & 0 & 0 & 0 & 1 & 1 & 0 & 0 & 0 & 0 & 0 & 0 & 0 & 0 & 0 & 0 \\
   0 & -1 & 0 & 0 & 0 & 0 & 0 & 0 & 1 & 1 & 0 & 0 & 0 & 0 & 0 & 0 & 0 & 0 \\
   0 & 0 & -1 & 0 & 0 & 0 & 1 & 0 & 1 & 0 & 0 & 0 & 0 & 0 & 0 & 0 & 0 & 0 \\
   0 & 0 & 0 & -1 & 0 & 0 & 0 & 1 & 0 & 1 & 0 & 0 & 0 & 0 & 0 & 0 & 0 & 0 \\
   0 & 0 & -1 & 0 & 0 & 0 & 0 & 0 & 0 & 0 & 1 & 1 & 0 & 0 & 0 & 0 & 0 & 0 \\
   0 & 0 & 0 & -1 & 0 & 0 & 0 & 0 & 0 & 0 & 0 & 0 & 1 & 1 & 0 & 0 & 0 & 0 \\
   0 & 0 & 0 & 0 & -1 & 0 & 0 & 0 & 0 & 0 & 1 & 0 & 1 & 0 & 0 & 0 & 0 & 0 \\
   0 & 0 & 0 & 0 & 0 & -1 & 0 & 0 & 0 & 0 & 0 & 1 & 0 & 1 & 0 & 0 & 0 & 0 \\
   0 & 0 & 0 & 0 & -1 & 0 & 0 & 0 & 0 & 0 & 0 & 0 & 0 & 0 & 1 & 1 & 0 & 0 \\
   0 & 0 & 0 & 0 & 0 & -1 & 0 & 0 & 0 & 0 & 0 & 0 & 0 & 0 & 0 & 0 & 1 & 1 \\
   -1 & 0 & 0 & 0 & 0 & 0 & 0 & 0 & 0 & 0 & 0 & 0 & 0 & 0 & 1 & 0 & 1 & 0 \\
   0 & -1 & 0 & 0 & 0 & 0 & 0 & 0 & 0 & 0 & 0 & 0 & 0 & 0 & 0 & 1 & 0 & 1
   \end{pmatrix}
\end{equation}
with the upper bound (12-component) vector:
\[
  b = 
  \begin{pmatrix}
    1 \\
    1 \\
    1 \\
    0 \\
    \vdots \\
    0
  \end{pmatrix}
\]

The constraint is thus:
\[
  A \mu \leq b
\]

We use the given potentials to calculate $\mu$ for $\beta \in \{-1,1\}$.

The output for \textit{beta} = +1.0
\begin{verbatim}
  Optimization terminated successfully.
        Current function value: 1.100000    
        Iterations: 13
  beta= 1.0
  solution vector mu=
  [ 1.  0.  1.  0.  1.  0.  1.  0.  0.  0.  1.  0.  0.  0.  1.  0.  0.  0.]
  result: [ 0.  0.  0.]
\end{verbatim}

The output for \textit{beta} = -1.0
\begin{verbatim}
  Optimization terminated successfully.
          Current function value: -1.900000   
          Iterations: 16
  beta= -1.0
  solution vector mu=
  [ 0.5  0.5  0.5  0.5  0.5  0.5  0.   0.5  0.5  0.   0.   0.5  0.5  0.   0.
  0.5  0.5  0. ]
  result: [ 0.  0.  0.]
\end{verbatim}

We see that the program gives a non-integer result for $\beta=-1$.


\lstinputlisting[caption=exercise3.py,label=lst:lp]{exercise3.py}

\end{document}

