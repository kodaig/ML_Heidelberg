\documentclass[12pt]{scrartcl}

%%%%%
% Header for solutions for the course Machine Learning for Computer Vision
% Summer 2017
%%%%%

\usepackage[utf8]{inputenc}
\usepackage[english]{babel}

\usepackage{
  amsmath,
  graphicx,
  tabularx,
  caption,
  subcaption,
  float,
  listingsutf8,
}

\usepackage[load-configurations = abbreviations]{siunitx}
\usepackage{hyperref}
\usepackage[english]{cleveref}

%%%%% format settings
%------------
% \setlength{\parindent}{0pt} %no indent
%------------

%%%%% settings for listings
\lstset{
  language=python,
  basicstyle=\footnotesize,  % font size
  showspaces=false,
  showstringspaces=false,
  frame=single, %tb
  breaklines=true,
  % backgroundcolor=\color[RGB]{245,245,244},
  % otherkeywords={self},             % Add keywords here
  % keywordstyle=\color{blue},
  % commentstyle=\it\color[RGB]{0,96,96}\ttfamily,
  % stringstyle=\color[RGB]{255,140,0},
  % numbers=left,
  % stepnumber=5,
}

%%% commands
%------------
\newcommand{\code}{\texttt}
%------------


\usepackage{physics}

\author{Kodai Matsuoka, Yuyan Li}
\subject{Machine Learning for Computer Vision}
\title{Exercise 11}
% \subtitle{}
\date{July 14, 2017}


\begin{document}

\maketitle

Our program is found in \cref{lst:code}

\section{Forward Pass}

Our neural network is given by

\[
  z = sigmoid(\sin(\phi) \cdot x_0 + \cos(\phi) \cdot x_1 + r).
\]

We do a binary classification with the probabilities

\[ p(y=0)=z \quad \text{and} \quad p(y=1)=1-z \]



\section{Fischer Matrix}

The entries of the Fischer matrix are given by

\[
  F_{a,b} = \sum_{x \in X} \sum_{y \in \{0,1\}} \left( \pdv{}{a} \log p(y|x) \right) \left( \pdv{}{b} \log p(y|x) \right) p(y|x)
\]

We need the derivatives of $z(x;\phi,r)$ with respect to $\phi$ and $r$.

\[
  \pdv{z}{\phi} = \left( x_0 \cos(\phi) - x_1 \sin(\phi) \right) \cdot sigmoid'(\sin(\phi) \cdot x_0 + \cos(\phi) \cdot x_1 + r)
\]

\[
  \pdv{z}{r} = sigmoid'(\sin(\phi) \cdot x_0 + \cos(\phi) \cdot x_1 + r)
\]

The derivative of the sigmoid is given by:
\[
  sigmoid'(x) = sigmoid(x) \cdot (1 - sigmoid(x))
\]

With these we can compute the Fischer matrix entries:

\[
  F_{\phi \phi} = \sum_{x \in X} \frac{\pdv{z}{\phi}^2}{z(1-z)}
\]
\[
  F_{rr} = \sum_{x \in X} \frac{\pdv{z}{r}^2}{z(1-z)}
\]
\[
  F_{\phi r} = \sum_{x \in X} \frac{\left( \pdv{z}{\phi} \pdv{z}{r} \right)^2}{z(1-z)}
\]

\begin{figure}[h]
  \centering
  \includegraphics[width=.49\linewidth]{heatmaps_a}
  \includegraphics[width=.49\linewidth]{heatmaps_b}
  \caption{The heatmaps show the values of the loss and the Fischer matrix entries of the two datasets A (left) and B (right). The plots are made with N=M=50 and a sample size for each class of $N_{data}=100$}
  \label{fig:hm}
\end{figure}


\clearpage
\lstinputlisting[caption=fischer.py,label=lst:code]{fischer.py}

\end{document}


